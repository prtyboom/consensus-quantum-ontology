\section{Слабополевой предел и оптические эффекты}

\subsection{Параметризованная пост-ньютоновская метрика}

Для связи нашего скалярного поля $\varepsilon(x)$ с общей теорией относительности мы используем стандартный формализм параметризованной пост-ньютоновской (PPN) метрики \cite{Will2014}. Это позволяет избежать эвристических конструкций (типа «эффективного показателя преломления») и опереться на проверенный релятивистский аппарат.

\subsubsection{PPN-формализм в слабом поле}

В слабом гравитационном поле ($|GM/(c^2r)| \ll 1$) метрику пространства-времени можно записать в виде:
\begin{equation}\label{eq:ppn_metric}
\begin{cases}
g_{00} = -(1 + 2\Phi/c^2 + \mathcal{O}(\Phi^2/c^4)) \\[0.2cm]
g_{ij} = (1 - 2\gamma\Phi/c^2)\delta_{ij} + \mathcal{O}(\Phi^2/c^4) \\[0.2cm]
g_{0i} = \mathcal{O}(v/c) \quad \text{(пренебрегаем)}
\end{cases}
\end{equation}

где:
\begin{itemize}
    \item $\Phi(x)$ — ньютоновский гравитационный потенциал
    \item $\gamma$ — PPN-параметр (для ОТО $\gamma = 1$)
    \item $\delta_{ij}$ — метрика плоского пространства
\end{itemize}

В нашем фреймворке:
\begin{equation}\label{eq:phi_epsilon_reminder}
\Phi = -c^2(1 - \varepsilon) = -c^2\delta
\end{equation}

где $\delta = 1 - \varepsilon$ — отклонение от Абсолюта.

\subsubsection{Выбор параметра $\gamma$}

Мы принимаем \textbf{$\gamma = 1$} (значение ОТО), что согласуется с экспериментальными ограничениями \cite{Bertotti2003}:
\begin{equation}
|\gamma - 1| < 2.3 \times 10^{-5} \quad \text{(Cassini, 2003)}
\end{equation}

При $\gamma = 1$ метрика~\eqref{eq:ppn_metric} принимает вид:
\begin{equation}\label{eq:ppn_gamma1}
ds^2 = -\left(1 + \frac{2\Phi}{c^2}\right)c^2dt^2 + \left(1 - \frac{2\Phi}{c^2}\right)(dx^2 + dy^2 + dz^2)
\end{equation}

Подставляя $\Phi = -c^2(1-\varepsilon)$:
\begin{equation}
ds^2 = -(2\varepsilon - 1)c^2dt^2 + (2 - \varepsilon)(dx^2 + dy^2 + dz^2) + \mathcal{O}(\delta^2)
\end{equation}

\begin{remark}
Мы \textbf{не} вводим «эффективный показатель преломления» $n = 2 - \varepsilon$, как в некоторых эвристических подходах. Вместо этого используем метрику~\eqref{eq:ppn_gamma1} и стандартные формулы геодезического движения и распространения света.
\end{remark}

\subsection{Гравитационное линзирование}

\subsubsection{Вывод угла отклонения}

Рассмотрим световой луч, проходящий на прицельном расстоянии $b$ от точечной массы $M$. Используем геодезическое уравнение для нулевых геодезических ($ds^2 = 0$).

В слабом поле угол отклонения определяется интегралом вдоль невозмущённой траектории \cite{Weinberg1972}:
\begin{equation}
\theta = -\frac{2}{c^2}\int_{-\infty}^{+\infty} \frac{\partial\Phi}{\partial r_\perp} dl
\end{equation}

где $r_\perp$ — расстояние от массы до точки на траектории, $l$ — параметр вдоль прямой.

Для точечной массы $\Phi(r) = -GM/r$:
\begin{equation}
\frac{\partial\Phi}{\partial r_\perp} = -\frac{GM}{r^2} \cdot \frac{b}{r} = -\frac{GMb}{r^3}
\end{equation}

где $r = \sqrt{l^2 + b^2}$. Интегрируя по $l$:
\begin{equation}
\theta = \frac{2GM}{c^2b} \int_{-\infty}^{+\infty} \frac{dl}{(l^2 + b^2)^{3/2}}
\end{equation}

Стандартный интеграл:
\begin{equation}
\int_{-\infty}^{+\infty} \frac{dl}{(l^2 + b^2)^{3/2}} = \frac{2}{b^2}
\end{equation}

откуда:
\begin{equation}\label{eq:deflection_angle}
\boxed{\theta = \frac{4GM}{c^2b}}
\end{equation}

\subsubsection{Выражение через поле $\varepsilon$}

Из $\Phi = -c^2(1-\varepsilon)$ следует:
\begin{equation}
\nabla\Phi = c^2\nabla\varepsilon
\end{equation}

Формула линзирования~\eqref{eq:deflection_angle} может быть переписана:
\begin{equation}
\theta = -\frac{2}{c^2}\int \frac{\partial\Phi}{\partial r_\perp} dl = 2\int \frac{\partial\varepsilon}{\partial r_\perp} dl
\end{equation}

Для точечной массы $\varepsilon(r) = 1 - GM/(c^2r)$:
\begin{equation}
\frac{\partial\varepsilon}{\partial r_\perp} = \frac{GM}{c^2} \cdot \frac{b}{r^3}
\end{equation}

что воспроизводит~\eqref{eq:deflection_angle}.

\subsubsection{Экспериментальная проверка}

\begin{enumerate}
    \item \textbf{Эддингтон, 1919} \cite{Dyson1920}: Отклонение света Солнца, $M = M_\odot$, $b = R_\odot$:
    \begin{equation}
    \theta_\odot = \frac{4GM_\odot}{c^2 R_\odot} = 1.75'' \quad \text{(измерено: } 1.98'' \pm 0.16''\text{)}
    \end{equation}
    
    \item \textbf{VLBI, 1995–2024} \cite{Fomalont2009}: Радиоинтерферометрия со сверхдлинными базами:
    \begin{equation}
    \theta_{\text{измер}}/\theta_{\text{ОТО}} = 0.99992 \pm 0.00014 \quad \text{(точность 0.01\%)}
    \end{equation}
    
    \item \textbf{Гравитационные линзы}: Кольца Эйнштейна, дуги, множественные изображения — тысячи примеров (HST, JWST).
\end{enumerate}

Наш фреймворк с $\varepsilon$-полем воспроизводит эти результаты точно.

\subsection{Задержка Шапиро}

\subsubsection{Вывод временной задержки}

Рассмотрим распространение света (радиосигнала) между двумя точками на расстояниях $r_1$ и $r_2$ от массы $M$ с прицельным параметром $b$.

Из метрики~\eqref{eq:ppn_gamma1} при $ds^2 = 0$:
\begin{equation}
c^2dt^2 = \frac{1 - 2\Phi/c^2}{1 + 2\Phi/c^2} dl^2 \approx \left(1 - \frac{4\Phi}{c^2}\right) dl^2
\end{equation}

где $dl$ — пространственный элемент. Эффективная скорость света:
\begin{equation}
v_{\text{eff}} = \frac{dl}{dt} \approx c\left(1 + \frac{2\Phi}{c^2}\right) = c\left(1 - 2(1-\varepsilon)\right) = c(2\varepsilon - 1)
\end{equation}

Задержка относительно плоского пространства:
\begin{equation}
\Delta t = \int \left(\frac{1}{v_{\text{eff}}} - \frac{1}{c}\right) dl = -\frac{2}{c^2}\int \Phi(l) \, dl
\end{equation}

Для точечной массы $\Phi(r) = -GM/r$, где $r = \sqrt{l^2 + b^2}$:
\begin{equation}
\Delta t = \frac{2GM}{c^3} \int_{-L}^{+L} \frac{dl}{\sqrt{l^2 + b^2}}
\end{equation}

Интегрируя:
\begin{equation}
\int \frac{dl}{\sqrt{l^2 + b^2}} = \ln\left(l + \sqrt{l^2 + b^2}\right) + \text{const}
\end{equation}

В пределе $L \gg b$ (лучи от Земли до космического аппарата, проходящие около Солнца):
\begin{equation}\label{eq:shapiro_delay}
\boxed{\Delta t = \frac{4GM}{c^3} \ln\frac{4r_1 r_2}{b^2}}
\end{equation}

\subsubsection{Экспериментальная проверка}

\begin{enumerate}
    \item \textbf{Cassini, 2003} \cite{Bertotti2003}: Радиосигнал Земля–Кассини около Солнца:
    \begin{equation}
    \Delta t_{\text{измер}}/\Delta t_{\text{ОТО}} = 1.00001 \pm 0.00001 \quad \text{(точность 0.001\%)}
    \end{equation}
    
    \item \textbf{Двойные пульсары} \cite{Kramer2006}: PSR J0737–3039, точность до микросекунд.
\end{enumerate}

\subsection{Гравитационное красное смещение}

\subsubsection{Вывод из метрики}

Для фотона, испущенного в точке с потенциалом $\Phi_1$ и принятого в точке $\Phi_2$, сохранение энергии на геодезической даёт:
\begin{equation}
\frac{E_1}{\sqrt{|g_{00}(r_1)|}} = \frac{E_2}{\sqrt{|g_{00}(r_2)|}}
\end{equation}

Из метрики~\eqref{eq:ppn_gamma1}: $g_{00} = -(1 + 2\Phi/c^2)$, откуда:
\begin{equation}
\frac{\omega_1}{\omega_2} = \sqrt{\frac{1 + 2\Phi_2/c^2}{1 + 2\Phi_1/c^2}} \approx 1 + \frac{\Phi_2 - \Phi_1}{c^2}
\end{equation}

Для фотона, выходящего из гравитационного колодца ($\Phi_1 < 0$, $\Phi_2 = 0$):
\begin{equation}
\frac{\Delta\omega}{\omega} = \frac{\Phi_1}{c^2} = -(1 - \varepsilon_1)
\end{equation}

или
\begin{equation}\label{eq:redshift}
\boxed{\frac{\omega(\infty)}{\omega(r)} = \varepsilon(r) = 1 - \frac{GM}{c^2r}}
\end{equation}

\subsubsection{Экспериментальная проверка}

\begin{enumerate}
    \item \textbf{Паунд–Ребка, 1959} \cite{Pound1960}: Мёссбауэровское красное смещение в башне высотой $h = 22.5$ м:
    \begin{equation}
    \frac{\Delta\omega}{\omega} = \frac{gh}{c^2} = 2.46 \times 10^{-15} \quad \text{(точность 10\%)}
    \end{equation}
    
    \item \textbf{Gravity Probe A, 1976} \cite{Vessot1980}: Ракетный эксперимент на высоте 10,000 км:
    \begin{equation}
    \Delta\omega/\omega \text{ измерено с точностью } 7 \times 10^{-5}
    \end{equation}
    
    \item \textbf{GPS} \cite{Ashby2003}: Спутники на высоте 20,200 км испытывают сдвиг $\sim 45$ мкс/день, который полностью учитывается в системе.
\end{enumerate}

Все результаты согласуются с формулой~\eqref{eq:redshift}.

\subsection{Почему не показатель преломления}

Некоторые эвристические подходы вводят «эффективный показатель преломления гравитационного поля» $n = 2 - \varepsilon$. Мы избегаем этой конструкции по следующим причинам:

\begin{enumerate}
    \item \textbf{Нестрогость}: Понятие показателя преломления применимо к волнам в среде, а не к геометрии пространства-времени.
    
    \item \textbf{Противоречия}: Формула $n = 2 - \varepsilon$ даёт $n(\varepsilon=1) = 1$ (правильно), но $n(\varepsilon=0) = 2$, что не имеет ясного физического смысла в контексте метрики.
    
    \item \textbf{Избыточность}: PPN-метрика~\eqref{eq:ppn_gamma1} — стандартный, проверенный инструмент. Все оптические эффекты корректно выводятся из геодезических без дополнительных предположений.
    
    \item \textbf{Путаница размерностей}: Показатель преломления безразмерен, но его связь с метрикой $g_{\mu\nu}$ требует постулатов о том, какая компонента метрики соответствует $n$.
\end{enumerate}

\textbf{Вывод:} Мы используем $\varepsilon$-поле для построения метрики через $\Phi = -c^2(1-\varepsilon)$, а затем применяем стандартную ОТО для вывода наблюдаемых.

\subsection{Резюме раздела}

Мы показали:

\begin{enumerate}
    \item PPN-метрика с $\gamma = 1$ и $\Phi = -c^2(1-\varepsilon)$ согласуется с нашим вариационным фреймворком.
    
    \item Из этой метрики стандартными методами выводятся:
    \begin{itemize}
        \item Гравитационное линзирование: $\theta = 4GM/(c^2b)$
        \item Задержка Шапиро: $\Delta t = (4GM/c^3)\ln(4r_1r_2/b^2)$
        \item Красное смещение: $\omega(\infty)/\omega(r) = \varepsilon(r)$
    \end{itemize}
    
    \item Все три эффекта проверены экспериментально с точностью от 0.001\% (Cassini) до 0.01\% (VLBI).
    
    \item Наш фреймворк воспроизводит их без свободных параметров и без эвристических конструкций типа $n = 2 - \varepsilon$.
\end{enumerate}

Эти результаты демонстрируют, что консенсусная онтология (разделы 2–3) не только восстанавливает ньютоновскую гравитацию, но и согласуется с релятивистскими оптическими тестами ОТО в слабом поле.

В следующем разделе мы представим детальную численную валидацию уравнения Пуассона~\eqref{eq:poisson_epsilon} с тестами сходимости, изотропии и линейности.