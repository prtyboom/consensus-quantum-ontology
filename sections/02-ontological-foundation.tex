\section{Онтологический фундамент}

\subsection{Абсолют как референтное состояние максимальной энтропии}

В основе нашей онтологии лежит концепция \textit{Абсолюта} — референтного состояния, которое мы обозначаем скалярным полем $\varphi(x) \equiv 1$. Это не произвольная нормировка, а операциональное определение, вытекающее из фундаментальных принципов квантовой статистической механики.

\begin{postulate}[Абсолют как максимум энтропии]
Состояние Абсолюта $\varphi = 1$ определяется как квантовое состояние с максимальной энтропией фон Неймана при фиксированной полной энергии вселенной:
\begin{equation}
S_{vN}[\rho] = -k_B \mathrm{Tr}(\rho \ln \rho) \to \max
\end{equation}
при ограничении $\mathrm{Tr}(\rho \hat{H}) = E_{\text{total}} = \text{const}$.
\end{postulate}

\begin{proposition}
Решением задачи максимизации энтропии фон Неймана при фиксированной энергии является состояние максимальной смешанности:
\begin{equation}
\rho_{\max} \propto \mathbb{1}
\end{equation}
где $\mathbb{1}$ — единичный оператор в гильбертовом пространстве.
\end{proposition}

\begin{proof}
Используем метод множителей Лагранжа. Функционал:
\begin{equation}
\mathcal{L}[\rho] = -k_B \mathrm{Tr}(\rho \ln \rho) - \lambda[\mathrm{Tr}(\rho \hat{H}) - E_{\text{total}}] - \mu[\mathrm{Tr}(\rho) - 1]
\end{equation}

Стационарность $\delta\mathcal{L}/\delta\rho = 0$ даёт:
\begin{equation}
-k_B(\ln\rho + 1) - \lambda \hat{H} - \mu = 0
\end{equation}

откуда
\begin{equation}
\rho = \exp\left(-\frac{\lambda}{k_B}\hat{H} - \frac{\mu + k_B}{k_B}\right)
\end{equation}

В пределе $\lambda \to 0$ (бесконечная температура, полное перемешивание):
\begin{equation}
\rho \to \frac{1}{Z}\mathbb{1}, \quad Z = \mathrm{Tr}(\mathbb{1})
\end{equation}

Это состояние соответствует максимальной энтропии $S_{vN} = k_B \ln \dim(\mathcal{H})$.
\end{proof}

\textbf{Физическая интерпретация:} Абсолют ($\varphi = 1$) — это состояние полной симметрии, где все микросостояния равновероятны, нет выделенных направлений, локализации или дифференциации. Это не «пустота» (которая соответствовала бы $\varphi = 0$), а \textit{недифференцированная полнота} — состояние, содержащее все потенциальные возможности в равной мере.

Нормировка $\mathrm{Tr}(\rho_{\max}) \equiv 1$ фиксирует значение $\varphi = 1$.

\begin{remark}
Это отличается от КТП-вакуума $|0\rangle$, который имеет нулевую энергию, но определённую структуру (энергия нулевых колебаний, нарушение симметрий). Абсолют — это термодинамический максимум, не квантовое основное состояние.
\end{remark}

\subsection{Субтракция как дифференциация}

Материальные состояния возникают как \textit{отклонения} от Абсолюта:
\begin{equation}
\varphi(x) = 1 - \delta(x), \quad \delta(x) \geq 0
\end{equation}

где $\delta(x)$ — безразмерная мера «дефицита» относительно референтного состояния.

\textbf{Ключевая идея:} Материя — это НЕ «добавление чего-то к пустоте», а \textit{субтракция из полноты}, локальная дифференциация, нарушение максимальной симметрии. Если Абсолют — это «Океан» недифференцированной потенциальности, то материя — это «Солёный Человечек», локальное выделение определённости из неопределённости.

Математически, дифференциация означает снижение локальной энтропии:
\begin{equation}
S_{vN}[\rho(x)] < S_{vN}[\rho_{\max}] \quad \iff \quad \varphi(x) < 1
\end{equation}

Присутствие массы создаёт \textit{информационную структуру} — локализацию в пространстве, выделенные квантовые числа, определённую волновую функцию — что соответствует отклонению от состояния максимальной энтропии.

\begin{remark}
Эта онтология инвертирует стандартную картину: не «частицы существуют в пустоте», а «пустота (Абсолют) фундаментальна, а частицы — локальные нарушения её симметрии».
\end{remark}

\subsection{Консенсусное поле: строгая формулировка}

Ключевым объектом нашей теории является \textit{консенсусное поле} $\rho_C(x)$ — оператор плотности, представляющий коллективное квантовое состояние всех материальных систем в данной точке пространства.

\subsubsection{Определение со сглаживанием}

Прямое определение $\rho_C(x) = \sum_i (m_i/|x-x_i|^2)|\psi_i\rangle\langle\psi_i|$ страдает от:
\begin{enumerate}
    \item Сингулярности при $x \to x_i$
    \item Некорректной нормировки ($\mathrm{Tr}(\rho_C) \neq 1$)
    \item Неопределённости размерности
\end{enumerate}

Вводим \textbf{строгое определение} со сглаживающим ядром:

\begin{postulate}[Консенсусное поле]
Для набора квантовых систем с состояниями $|\psi_i\rangle$ (где $i$ нумерует все частицы/системы) и массами $m_i$, консенсусное поле определяется как:
\begin{equation}\label{eq:rho_c}
\rho_C(x, t) = \sum_{i=1}^{N} m_i K_\ell(x - x_i(t)) |\psi_i(t)\rangle\langle\psi_i(t)|
\end{equation}

где $K_\ell(r)$ — сглаживающее ядро с характерной шириной $\ell$ и нормировкой:
\begin{equation}
\int_{\mathbb{R}^3} K_\ell(r) \, d^3r = 1
\end{equation}
\end{postulate}

\textbf{Типичный выбор ядра} (гауссово):
\begin{equation}
K_\ell(r) = \frac{1}{(2\pi\ell^2)^{3/2}} \exp\left(-\frac{|r|^2}{2\ell^2}\right)
\end{equation}

где масштаб сглаживания $\ell$ определяется физикой задачи (например, $\ell \sim$ атомный масштаб для твёрдых тел, $\ell \sim$ комптоновская длина для элементарных частиц).

\subsubsection{Разложение на интенсивность и нормированное состояние}

Операторная плотность $\rho_C(x)$ не является стандартной матрицей плотности ($\mathrm{Tr}(\rho_C) \neq 1$). Вводим разложение:
\begin{align}
A_C(x) &= \mathrm{Tr}\,\rho_C(x) \geq 0 \quad \text{(локальная интенсивность)} \label{eq:A_C}\\
\sigma_C(x) &= \frac{\rho_C(x)}{A_C(x)}, \quad \mathrm{Tr}\,\sigma_C(x) = 1 \quad \text{(нормированная матрица плотности)} \label{eq:sigma_C}
\end{align}

\textbf{Физический смысл:}
\begin{itemize}
    \item $A_C(x)$ — «плотность консенсуса», имеет размерность $[\text{масса}]$ (интегральная мера присутствия материи)
    \item $\sigma_C(x)$ — «локальное квантовое состояние консенсуса», нормированная матрица плотности в точке $x$
    \item Эрмитовость: $\rho_C^\dagger = \rho_C$ (следует из $|\psi_i\rangle\langle\psi_i|^\dagger = |\psi_i\rangle\langle\psi_i|$)
    \item Положительность: $\rho_C \geq 0$ (сумма положительных операторов)
\end{itemize}

\subsubsection{Связь с полем $\varepsilon$}

Поле $\varepsilon(x)$ связано с консенсусом через монотонную функцию:
\begin{equation}
\delta(x) = f(A_C(x))
\end{equation}

где $f: \mathbb{R}_+ \to [0,1)$ — возрастающая функция с $f(0) = 0$ (вакуум $\to$ $\delta=0$, $\varepsilon=1$).

\textbf{Простейший выбор} (линейная связь в слабом поле):
\begin{equation}
\delta(x) = \frac{\kappa}{4\pi} A_C(x) \quad \text{для } A_C \ll 1/\kappa
\end{equation}

Конкретный вид $f$ — предмет будущих исследований; в данной работе мы используем линейное приближение и связываем $\varepsilon$ с классической плотностью массы $\rho(x) = \sum_i m_i \delta^3(x - x_i)$ через уравнение Пуассона (Раздел 3).

\subsection{Решение проблемы циркулярности}

\subsubsection{Постановка проблемы}

Если масса определяется как
\begin{equation}
m_i = \alpha \cdot \mathrm{Tr}(|\psi_i\rangle\langle\psi_i| \cdot \rho_C)
\end{equation}

а консенсусное поле зависит от масс $\rho_C = \sum_j m_j K_\ell|\psi_j\rangle\langle\psi_j|$, возникает \textit{циркулярная зависимость}: масса определяет консенсус, консенсус определяет массу.

\subsubsection{Итеративная самосогласованность}

Мы разрешаем эту проблему через \textit{итеративную процедуру самосогласования}:

\begin{equation}\label{eq:selfconsistent}
\begin{cases}
m_i^{(n+1)} = m_i^{\text{bare}} + \alpha \cdot \mathrm{Tr}\left(|\psi_i\rangle\langle\psi_i| \cdot \rho_C^{(n)}\right) \\[0.3cm]
\rho_C^{(n+1)}(x) = \sum_j m_j^{(n+1)} K_\ell(x - x_j) |\psi_j\rangle\langle\psi_j|
\end{cases}
\end{equation}

где:
\begin{itemize}
    \item $m_i^{\text{bare}}$ — «голая» масса (барионная масса из КХД, или измеренная масса в стандартной физике)
    \item $\alpha$ — малый безразмерный параметр ($\alpha \ll 1$)
    \item $n = 0, 1, 2, \ldots$ — номер итерации
\end{itemize}

\textbf{Начальное условие:} $m_i^{(0)} = m_i^{\text{bare}}$ (стандартная масса).

\begin{lemma}[Сходимость самосогласованности]
При условии $\alpha \ll m_i^{\text{bare}}/\langle A_C \rangle$ итерационная схема~\eqref{eq:selfconsistent} сходится к неподвижной точке:
\begin{equation}
m_i^* = m_i^{\text{bare}} + \delta m_i
\end{equation}

где консенсусная поправка
\begin{equation}
|\delta m_i| \sim \alpha \langle A_C \rangle \ll m_i^{\text{bare}}
\end{equation}
\end{lemma}

\begin{proof}[Набросок доказательства]
Определим оператор итерации $\mathcal{F}: m^{(n)} \mapsto m^{(n+1)}$. В линейном приближении:
\begin{equation}
m_i^{(n+1)} - m_i^* = \alpha \sum_j (m_j^{(n)} - m_j^*) \cdot T_{ij}
\end{equation}

где $T_{ij} = \int K_\ell(x-x_j) \langle\psi_i|\psi_j\rangle^2 d^3x$.

Оператор $\mathcal{T} = \alpha \cdot T$ имеет норму $\|\mathcal{T}\| \sim \alpha N$ (где $N$ — число частиц). При $\alpha \ll 1/N$ оператор — сжимающее отображение, итерации сходятся геометрически.

Полное доказательство требует анализа собственных значений $T_{ij}$ и выходит за рамки данной работы.
\end{proof}

\textbf{Физическая интерпретация:} Измеренная масса частицы состоит из двух компонент:
\begin{equation}
m_i^{\text{measured}} = m_i^{\text{bare}} + m_i^{\text{consensual}}
\end{equation}

\begin{itemize}
    \item $m_i^{\text{bare}}$ — внутренняя масса (массы кварков, энергия связи глюонов, взаимодействие с полем Хиггса)
    \item $m_i^{\text{consensual}}$ — вклад от согласования с окружающим консенсусом
\end{itemize}

Для барионов $m_i^{\text{consensual}}/m_i^{\text{bare}} \sim \alpha \sim 10^{-6}$ (оценка), что находится ниже текущего предела точности масс-спектрометрии ($\sim 10^{-9}$ для атомных масс).

\subsection{Эффективная гравитационная связь}

Вместо модификации \textit{массы} (что нарушило бы принцип эквивалентности), мы вводим \textit{эффективную связь с гравитационным полем}:

\begin{equation}\label{eq:eff_coupling}
g_i^{\text{eff}} = g \left[1 + \alpha \cdot \mathrm{Tr}\left(|\psi_i\rangle\langle\psi_i| \cdot \sigma_C(x_i)\right)\right]
\end{equation}

где $g$ — гравитационное ускорение, $\sigma_C$ — нормированная матрица плотности консенсуса~\eqref{eq:sigma_C}.

\textbf{Ключевое свойство:} При $\alpha \ll 1$:
\begin{equation}
\left|\frac{g_i^{\text{eff}} - g}{g}\right| \sim \alpha \ll 10^{-13}
\end{equation}

что согласуется с тестами принципа эквивалентности Этвёша ($\eta < 10^{-13}$) \cite{Eotvos2008}.

\textbf{Вывод:} Инерционная масса $m_i^{\text{inertial}} = m_i^{\text{gravitational}} = m_i^{\text{bare}} + \delta m_i$ остаётся одинаковой для всех взаимодействий, но локальная «чувствительность» к градиенту консенсуса $\nabla\rho_C$ может незначительно варьироваться в зависимости от квантового состояния $|\psi_i\rangle$.

\begin{remark}
Эффект~\eqref{eq:eff_coupling} вынесен в \textbf{Приложение B} как гипотеза, требующая экспериментальной проверки. Основная теория (разделы 3–6) использует только стандартную барионную массу $m_i^{\text{bare}}$ и не зависит от консенсусных поправок.
\end{remark}

\subsection{Резюме онтологии}

Мы ввели:

\begin{enumerate}
    \item \textbf{Абсолют} ($\varphi = 1$) — состояние максимальной энтропии фон Неймана, недифференцированная полнота.
    
    \item \textbf{Субтракция} ($\varphi < 1$) — материя как локальное отклонение от Абсолюта, дифференциация.
    
    \item \textbf{Консенсусное поле} $\rho_C(x) = \sum_i m_i K_\ell(x-x_i)|\psi_i\rangle\langle\psi_i|$ — коллективное квантовое состояние с корректной нормировкой.
    
    \item \textbf{Разложение} $\rho_C = A_C \cdot \sigma_C$, где $A_C$ — интенсивность, $\sigma_C$ — нормированная матрица плотности.
    
    \item \textbf{Самосогласованность} — итеративная схема для массы $m = m^{\text{bare}} + \delta m$ при малом $\alpha$.
    
    \item \textbf{Эффективная связь} $g_i^{\text{eff}}$ — не нарушает принцип эквивалентности при $\alpha \ll 10^{-13}$.
\end{enumerate}

Эта конструкция свободна от:
\begin{itemize}
    \item Сингулярностей (благодаря $K_\ell$)
    \item Проблем нормировки (разделение $A_C$ и $\sigma_C$)
    \item Циркулярности (самосогласованность)
    \item Конфликта с принципом эквивалентности (эффективная связь, не масса)
\end{itemize}

В следующем разделе мы покажем, как из этой онтологии \textit{вариационно} возникает уравнение Пуассона для гравитации.